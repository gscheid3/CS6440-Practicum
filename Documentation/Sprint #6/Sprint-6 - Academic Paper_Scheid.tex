\documentclass[letterpaper]{jdf}
\usepackage{etoolbox}
\makeatletter
\patchcmd{\@verbatim}
  {\verbatim@font}
  {\verbatim@font\small}
  {}{}
\makeatother

\pdfstringdefDisableCommands{ \def\\{} }

%\addbibresource{references.bib}

\author{Geoff Scheid}
\email{gscheid3@gatech.edu}
\title{CS 6440 - Intro to Health Informatics:\\Practicum Sprint 6}

\begin{document}

\maketitle

\begin{abstract}
    This document summarizes the final submission for the Crohn's Symptom Tracker web app developed for the Practicum Project in CS 6440.
    Background information is provided on Crohn's disease, the common issues patient's face with tracking their symptoms, 
    and the web app is proposed as a solution for those issues.
    The outcome of the web app development is discussed along with potential future development work to improve the web app in order to extend its use to more patients.
\end{abstract}

\section{Background}
Crohn's disease is a chronic inflammatory bowel disease that affects millions of people worldwide.
It is a condition that causes inflammation and irritation in the digestive tract, 
leading to symptoms such as abdominal pain, diarrhea, and fatigue. 
Crohn's disease is a complex condition that can have a significant impact on a person's quality of life, 
and managing it can be challenging.\\
Additional information on Crohn's disease my be found here:\\
https://www.crohnscolitisfoundation.org/what-is-crohns-disease

\section{Problem Statement}
One way to manage Crohn's disease is by tracking symptoms, 
which can help patients and healthcare providers better understand the disease and how it is affecting the patient. 
A web app designed for this purpose can make symptom tracking easier and more convenient for patients, 
and provide valuable insights into the condition for healthcare providers.\\
A template symptom tracker provided by the Crohn's and Colitis Foundation:\\
https://www.crohnscolitisfoundation.org/sites/default/files/legacy/assets/pdfs/ibd-symptom-tracker.pdf
\clearpage
\section{Solution}
The goal of this project is to create a web app that allows patients with Crohn's disease to easily track their symptoms, 
monitor their condition over time, and share their symptom data with their healthcare providers. 
The app will be designed with the patient's needs in mind, with a user-friendly interface and intuitive features. 
By empowering patients to take a more active role in managing their condition, 
the app can help improve their quality of life and overall health outcomes.

\section{Outcome}
The final submission of this web app accomplishes a portion of the scope in the original solution.
The web app as designed is able to track stool-related symptoms over time, visually display two important metrics,
and accessible by any computing device such that a patient can easily show their healthcare provider.
A user is able to create a new symptom with a date, stool type, bleeding indicator, pain level, and notes.
All symptoms are displayed in a Recent Symptoms list, and the user can view, modify, or delete an existing symptom from that list.
The dashboard graphs are populated from the user-provided data on stool type and pain level.

\section{Future Work}
The scope of this web app could be significantly increased to track more types of symptoms and add more data visualizations.
Additional symptoms such as appetite, weight gain or loss, and mood would also be useful to track, and visualizations of 
those data would help a patient understand how those symptoms are trending over time.
A user authorization service could be added to allow multiple users to create their own profiles and store their own data.
Additional user authorization could be implemented for healthcare providers to view their patient's data.
A reporting service could be added to produce reports by request or automatically for review by the patient and provider.
Reports could be generated with similar visualizations to demonstrate trends over time.

\end{document}